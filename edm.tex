\documentclass{amsbook}
\usepackage[a5paper, margin=2cm]{geometry}
\usepackage{microtype, amsthm, amssymb, palatino, euler}
\numberwithin{section}{chapter}
\theoremstyle{plain}
\newtheorem{thm}{Theorem}[section]
\newtheorem{prop}[thm]{Proposition}
\theoremstyle{definition}
\newtheorem{defn}[thm]{Definition}
\title{Elementary Discrete Mathematics}
\setlength\parindent{0pt}
\setlength\normalparindent{0pt}
\def\br{~\\[1em]}
\def\ZZ{\mathbf Z}
\def\id{\text{id}}
\def\ord{\text{ord}}
\def\eps{\varepsilon}
\begin{document}
\maketitle
These notes are largely a selection of passages\\
that were
more or less directly
copied from:
\begin{itemize}
\item Kenneth Rosen's\\
\textsl{Elementary Number Theory and its Applications},
\item 
Jerry Shurman's writeups,
\item 
and Paolo Aluffi's \textsl{Algebra: Notes from the Underground}.
\end{itemize}
Of course, MathSE and Wikipedia were also consulted.
\br
There being no clean digital
copy of Rosen's book, I wrote these notes.
\tableofcontents
\chapter{Integers}
The word \textit{integer} comes from the Latin
for ``intact'' or ``whole.''
\br
The integers are a collection of numbers -- 
a collection so special that entire subfields of
mathematics are devoted to understanding them.
\br
The integers include the positive integers,
\[1,\quad 2,\quad 3,\quad 4,\quad 5,\quad\dots\]
as well as the negative integers,
\[-1,\quad -2,\quad -3,\quad -4,\quad -5,\quad\dots\]
There is also an integer called 0 that
is neither positive nor negative,
thought of as a neutral element of the collection.
\br
All together, the postive integers, negative integers, and zero
form the collection of integers, which we will denote $\ZZ$.
\br
We will also
denote the collection of positive integers by $\ZZ^+$.
\section{Well-Ordering and Induction}
A fundamental fact about the integers is:
\br
\hspace*{2em}
\fbox{
\begin{minipage}{25em}
\textit{The Well-Ordering Principle.}
Every nonempty subset $X\subseteq\ZZ^+$
has a least element.
\end{minipage}
}
\br
It is logically equivalent to the following:
\br
\hspace*{2em}
\fbox{
\begin{minipage}{25em}
\textit{The Principle of Induction.}
If a subset $X\subseteq\ZZ^+$ satisfies $1\in X$
and ($n\in X\implies n + 1\in X$),
then $X = \ZZ^+$.
\end{minipage}
}
\br
\begin{proof}
Let $X$ be a subset of $\ZZ^+$ satisfying $1\in X$
and \[n\in X\implies n + 1\in X.\]
We proceed by contradiction: suppose $X\ne\ZZ^+$.
Then there is a positive integer not in $X$, i.e.
$\ZZ^+\setminus X$ is nonempty. Then $\ZZ^+\setminus X$
has a least element $n$. Note that $n\ne 1$, since $1\in X$.
Thus $n > 1$, and since $n$ is the least element not in $X$,
$n - 1$ must be in $X$. But by assumption,
$(n - 1) + 1 = n\in X$, contradicting our assumption
that $n\notin X$. This proves that the well-ordering principle
implies the principle of induction.
\br
Conversely, consider a nonempty subset $Y\subseteq\ZZ^+$. If $Y$
has just one element, then that element is the least element of $Y$.
Now suppose the well ordering
principle is true for all subsets of $\ZZ^+$ with $n$ elements,
and suppose $Y$ has $n + 1$ elements. Take $y\in Y$ and let $z$
be the least element of $Y\setminus y$. Then $\min(\{y, z\})$ is the
least element of $Y$. 
This proves that the principle of induction implies the well-ordering
principle.
\end{proof}


Also relevant is the following
variation on the principle of induction:
\br
\hspace*{2em}
\fbox{
\begin{minipage}{25em}
\textit{Strong Induction.}
If a subset $X\subseteq\ZZ^+$ satisfies $1\in X$
and 
\[1,\dots,n\in X\implies n + 1\in X,\]
then $X = \ZZ^+$.
\end{minipage}
}
\br
Despite looking like a stricter requirement,
strong induction is actually implied by the principle of induction.
\begin{proof}
Let $Y\subseteq\ZZ^+$ satisfy $1\in Y$ and
\[1,\dots,n\in Y\implies n + 1\in Y.\]
Let $X\subseteq\ZZ^+$ be the set of all positive integers $n$
such that all positive integers less than or equal to $n$ are in $Y$.
Then $1\in X$. Furthermore, if $n\in X$, then $n + 1\in X$. So then by
the principle of induction, $X =\ZZ^+$, which implies $Y = \ZZ^+$.
\end{proof}

A function is said to be \textit{defined recursively} if it is defined
at 1 and if there exists a rule for finding $f(n)$ in terms of
$f(1)$ through $f(n-1)$. By strong induction,
such functions are defined on all of $\ZZ^+$.

\pagebreak

The archetypal example of a recursively defined function is the
\textit{factorial function}, given by
\[n! = \begin{cases}1 &\text{if }n = 0\\
n\cdot(n - 1)!&\text{otherwise}\end{cases}\]
For example, $6! = 720$.
\br
Defined in terms of the factorial function are the
\textit{binomial coefficients},
\[{n\choose k} = \frac{n!}{k!(n-k)!}\]
A quick computation shows that
\[
{n\choose k} + {n\choose k + 1} = {n + 1\choose k + 1}.
\]
Also note that ${n\choose 0} = {n\choose n} = 1$.
\br
By these observations, binomial coefficients
are always integers.

\begin{thm}[Binomial theorem]
Let $a$ and $b$ be integers and $n$ a nonnegative integer.
Then
\[(a + b)^n = \sum_{k = 0}^n{n\choose k}a^kb^{n-k}.\]
\end{thm}

\begin{proof}
By induction. To see that the claim is true for $n = 0$, note that
\[(a + b)^0 = 1 = \sum_{k = 0}^0{0\choose k}a^kb^{0-k}.\]
\br
Now assume the claim is true for all integers $n\le m$. Then
\begin{align*}
&\hspace{-1em}(a + b)^{m + 1} = (a + b)^m(a + b)\\[0.5em]
&= \left(\sum_{k = 0}^m{m\choose k}a^kb^{m-k}\right)(a + b)
\qquad\text{by the inductive step}\\[0.5em]
&=  \left(\sum_{k = 0}^m{m\choose k}a^{k+1}b^{m-k}\right) + 
 \left(\sum_{k = 0}^m{m\choose k}a^kb^{m-k + 1}\right)\\[0.5em]
&=\left(\sum_{k = 0}^{m-1}{m\choose k}a^{k+1}b^{m-k}\right) + a^{m+1}+
 \left(\sum_{k = 1}^m{m\choose k}a^kb^{m-k + 1}\right)
+ b^{m+1}\\[0.5em]
&=  \left(\sum_{k = 1}^m{m\choose k-1}a^kb^{m-k+1}\right) + a^{m+1}+ 
 \left(\sum_{k = 1}^m{m\choose k}a^kb^{m-k + 1}\right)
+ b^{m+1}\\[0.5em]
&= a^{m+1}+  \left(\sum_{k = 1}^m\left({m\choose k-1}
+ {m\choose k}\right)a^kb^{m-k+1}\right) + b^{m+1}\\[0.5em]
&= a^{m+1}+  \left(\sum_{k = 1}^m{m + 1\choose k}a^kb^{m-k+1}\right)
+ b^{m+1}\\[0.5em]
&= \sum_{k = 0}^{m+1}{m + 1\choose k}a^kb^{m+1-k}.
\end{align*}
By induction, the claim is true for all nonnegative integers $n$.
\end{proof}
Two consequences of this formula are that
\[2^n = \sum_{k=0}^n{n\choose k}\quad\text{and}\quad
0 = \sum_{k=0}^n(-1)^k{n\choose k}.\]
\section{Divisibility}
The integers are closed under addition, subtraction,
and multiplication.
However, not every integer quotient forms another integer.
\begin{defn}
Let $a, b\in\ZZ$. We say $a$ \textit{divides} $b$
(or that $b$ \textit{is divisible by} $a$,
or that $b$ \textit{is a multiple of} $a$,
or that $a$ \textit{is a factor of} $b$) and write $a\mid b$
if there is 
some $c\in\ZZ$ such that $b = ac$.
\end{defn}
\begin{prop}
If $x\mid n$ and $x\mid m$, then for any integers $a$ and $b$,
\[x\mid (an + bm).\]
\end{prop}
\begin{proof}
We have $cx = n$ and $dx = m$ for some integers $c$ and $d$.
So
\[an + bm = acx + bdx = (ac + bd)x,\]
which implies $x\mid(an + bm)$.
\end{proof}
\begin{thm}[Division with remainder]
If $a$ and $b$ are integers such that $b > 0$, then
there exist unique integers $q$ and $r$ such that
\[a = bq + r\quad\text{and}\quad 0\le r < b.\]
\end{thm}
\begin{proof}
Define the \textit{floor} of $x$ (denoted $\lfloor x\rfloor$)
to be the largest integer
less than or equal to $x$.
Noting that
\[x - 1 < \lfloor x\rfloor\le x,\]
we set $q = \lfloor a/b\rfloor$, $r = a - b\lfloor a/b\rfloor$.
Now observe that
\[a/b - 1 < \lfloor a/b\rfloor\le a/b.\]
Multiplying through by $b$ yields
\[a - b < b\lfloor a/b\rfloor\le a.\]
Invert the inequality to obtain
\[-a\le -b\lfloor a/b\rfloor < b - a,\]
and then add $a$:
\[0\le a - b\lfloor a/b\rfloor < b.\]
To show $q$ and $r$ are unique, suppose we have $q'$ and $r'$
such that $a = bq' + r'$. Then $0 = b(q - q') + (r - r')$, i.e.
$b$ divides $r - r'$. But since $r$ and $r'$ are both between 0
and $b$, their difference is between $\pm b$, so $b$ can divide
$r - r'$ only if $r - r' = 0$, so we must have $r = r'$,
and $q = q'$ immediately after.
\end{proof}
\section{Prime Numbers}
The positive integer 1 has just one positive divisor. Every
other positive integer has at least two positive divisors,
being divisible by itself and 1.
\begin{defn}
A \textit{prime number} is a positive integer with exactly two
positive divisors. A positive integer with
more than two positive divisors is \textit{composite}.
\end{defn}

\pagebreak

\begin{prop}
Every integer greater than 1 has a prime divisor.
\end{prop}
\begin{proof}
By contradiction. Assume there is a positive integer $n$
greater than 1 with no prime divisors.
By the well-ordering principle we may take $n$ to be the smallest such
number. If an integer is prime,
it has a prime divisor (namely, itself).
Taking the contrapositive, an integer with no prime divisors must not
be prime. Hence, $n$ is not prime, so we may write $n = ab$ with
$1 < a < n$ and $1 < b < n$. Because $a < n$, $a$ must have a prime
divisor. But any prime divisor of $a$ must also be a prime divisor
of $n$, contradicting our assumption that $n$ had no prime divisors.
\end{proof}
\begin{thm}
There are infinitely many prime numbers.\footnotemark
\end{thm}
\footnotetext{
Consequently, 0 has infinitely many divisors, and is also 
the unique integer satisfying this condition.}
\begin{proof}
Consider
\[Q_n = n! + 1.\]
We know $Q_n$ has a prime divisor, which we will call $q_n$.
Observe that $q_n > n$: otherwise, we would have $q_n\le n$,
hence $q_n\mid n!$, hence $q_n\mid(Q_n - n!) = 1$, which is
impossible. We have thus found a prime larger than $n$
for any $n$, so there must be infinitely many primes.
\end{proof}
The gap between primes can be of any length.
Indeed, consider
\[(n+1)! + 2,\quad (n+1)! + 3,\quad\dots,\quad(n+1)! + n +1.\]
These $n$ consecutive integers are all composite.
\chapter{Coprimality and Factorization}
\section{Greatest Common Divisors}
\begin{defn}
We say an integer $d$ is a \textit{common divisor} of $a$ and $b$
if both $d\mid a$ and $d\mid b$, and that a common divisor is
\textit{greatest} if any common divisor $c$ of $a$ and $b$
also divides $d$. We denote by $(a, b)$ the greatest common divisor
of $a$ and $b$.
\end{defn}
\begin{thm}[Bezout's identity]
If $a$ and $b$ are integers not both 0, then $(a, b)$ is the smallest
positive linear combination of $a$ and $b$,
e.g. there are integers $m$ and $n$ such that
\[am + bn = (a, b).\]
\end{thm}
\begin{proof}
Consider all integer linear combinations of $a$ and $b$.
\br
Some of these linear
combinations are positive, such as $a^2 + b^2$, so the set of all
positive linear combinations of $a$ and $b$ is nonempty. By the
well-ordering principle this set has a least element, which we
will call $d$. Let $m$ and $n$ be such that $d = am + bn$.
\br
Use division with remainder to obtain
$a = dq + r$.
Note that
\[r = a - dq = a - (am + bn)q = a(1 - mq) - b(nq),\]
i.e. $r$ is a linear combination of $a$ and $b$. If $r$ were positive
then $d$ wouldn't be the smallest positive linear combination of $a$
and $b$, so $r = 0$, i.e. $d\mid a$.
A nearly identical argument shows that $d\mid b$.
\br
Suppose $c$ is a common divisor of $a$ and $b$. Then there exist
integers $u$ and $v$ such that $a = uc$ and $b = vc$.
But then
\[d = am + bn = ucm + vcn = (um + vn)c,\]
i.e. $c\mid d$. So $d = (a, b)$, and the proof is complete.
\end{proof}
\begin{defn}
We say two integers $a$ and $b$ are \textit{coprime} if
$(a, b) = 1$.
\end{defn}
\section{The Euclidean Algorithm}
Here is a way to compute greatest common divisors.
\br
\hspace*{2em}
\fbox{
\begin{minipage}{25em}
\textit{Euclidean Algorithm.}
Let $r_0 = a$ and $r_1 = b$ be nonnegative integers with $b\ne 0$.
Divide repeatedly to obtain
\[r_j = r_{j+1}q_{j+1} + r_{j+2},\qquad 0 < r_{j+2} < r_{j+1}\]
for $j\in\{0,\dots,n-2\}$. If $r_n = 0$, then $r_{n-1} = (a, b)$.
\end{minipage}
}
\br
We begin by showing that whenever
$a = bq + r$, we have $(a, b) = (b, r)$.
\begin{proof}
If both $c\mid a$ and $c\mid b$ then $c\mid a - bq = r$.
Also, if both $c\mid b$ and $c\mid r$ then $c\mid bq + r = a$.
Since the common divisors of $a$ and $b$ are the same as the
common divisors of $b$ and $r$, we have $(a, b) = (b, r)$.
\end{proof}
Now we show the Euclidean algorithm works.
\begin{proof}
In the situation described above, note that
\[(a, b) = (b, r_2) = (r_2, r_3) = \cdots = (r_{n-1}, 0) = r_{n-1}.\]
We hit 0 eventually because the sequence of remainders cannot contain
more than $|a|$ terms.
\end{proof}
\section{The Fundamental Theorem of Arithmetic}
\begin{thm}
Any positive integer can be uniquely factored into primes.
\end{thm}
First we prove existence by contradiction.
\begin{proof}
Let $n\in\ZZ^+$. Suppose $n$ were the least
positive integer such that $n$ cannot be factored into primes. Then
$n$ cannot itself be prime, so $n = ab$ with
$1 < a < n$ and $1 < b < n$. Thus, $a$ and $b$ admit
factorizations into primes. Combining these yields
a prime factorization of $n$, which contradicts our assumption that
$n$ had no such prime factorization.
\end{proof}
Before proving uniqueness, we need an auxillary fact.
\begin{prop}[Euclid's lemma]
If $a, b, c$ are
positive integers with $(a, b) = 1$ and $a\mid bc$, then $a\mid c$.
\end{prop}
\begin{proof}
Since $(a, b) = 1$, we may write $1 = am + bn$. Multiply by $c$
to obtain $c = amc + bnc$. But $a\mid amc$ and $a\mid bnc$,
so $a\mid c$.
\end{proof}

\pagebreak

Next, we need to show that primes do not decompose as factors.
\begin{prop}
If $a_1,\dots, a_n$ are integers and $p$ prime, 
\[p\mid a_1\cdots a_n\implies p\mid a_i\quad\text{for some }i.\]
\end{prop}
\begin{proof}
By induction. If $n = 1$, then $p = a_1$, hence $p\mid a_1$.
Now suppose the claim holds for $n = m$, and consider
$p = a_1\cdots a_{m+1}$. Then by what was just shown, either
$p\mid a_1\cdots a_m$ or $p\mid a_{m+1}$. But if $p\mid a_1\cdots a_m$
then $p\mid a_i$ for some $i$ by the inductive hypothesis.
\end{proof}
We are now ready to prove uniqueness of prime factorization.
\begin{proof}
Suppose $n$ is the smallest positive integer with
\[n = p_1\cdots p_s = q_1\cdots q_t\]
where the $p_i$ and $q_j$ are prime. Consider $p_1$. It must divide
one of the $q_i$, let's say $q_1$ without loss of generality.
But $q_1$
is prime, and since $p_1\ne 1$, we must have $p_1 = q_1$.
Divide through
by $p_1$ to obtain
\[n/p_1 = p_2\cdots p_s = q_2\dots q_t,\]
contradicting our assumption that $n$ was the smallest positive
integer with at least two prime factorizations.
\end{proof}
\chapter{Congruences}
The language of congruences was developed by Gauss.
\section{Basic Properties}
\begin{defn}
Let $a, b\in\ZZ$ and $m\in\ZZ^+$.
We say $a$ is \textit{congruent to} $b$ \textit{modulo} $m$
and write $a = b\pmod m$
if $m\mid(a - b)$.
\end{defn}
\begin{prop}
Being congruent modulo $m$ is an equivalence relation: 
it is reflexive, symmetric, and transitive.
\end{prop}
\begin{proof}
Since every number divides 0, we have $m\mid (a - a)$,
thus $a = a\pmod m$. Suppose $a = b\pmod m$. Then $m\mid(a - b)$,
hence $m\mid(b - a)$, hence $b = a\pmod m$. Finally, suppose
$a = b\pmod m$ and $b = c\pmod m$.
Then $m\mid (a - b)$ and $m\mid(b - c)$, hence
\[m\mid ((a - b) + (b - c)) = (a - c),\] hence $a = c\pmod m$.
\end{proof}
One can do arithmetic with congruences.
\begin{prop}
Let $a, b, c, d\in\ZZ$ and $m\in\ZZ^+$. If
$a = b\pmod m$ and $c = d\pmod m$, then
\begin{enumerate}
\item $a + c = b + d\pmod m$,
\item $a - c = b - d\pmod m$, and
\item $ac = bd\pmod m$.
\end{enumerate}
\end{prop}
\begin{proof}
We have $m\mid (a - b)$ and $m\mid (c - d)$.
Observe that
\[m\mid ((a - b) + (c - d)) = ((a + c) - (b + d)),\]
\[m\mid ((a - b) - (c - d)) = ((a - c) - (b - d)),\]
and
\[m\mid (a - b)c + b(c - d) = ac - bc + bc - bd = ac - bd,\]
from which the result follows.
\end{proof}
\section{Sun's Remainder Theorem}
\begin{thm}
Given integers $a_1,\dots,a_k$ and pairwise coprime integers
$n_1,\dots,n_k$, the system of congruences
\[x = a_i\pmod{n_i}\]
has a solution unique modulo $N = \prod_{i=1}^kn_i$.
\end{thm}
\begin{proof}
Let $N_i = N/n_i$. By pairwise coprimality of the $n_i$,
we have $(N_i, n_i) = 1$. Hence, we can find inverses $y_i$
such that $N_iy_i = 1\pmod{n_i}$.
Consider
\[x = a_1N_1y_1 + \cdots + a_kN_ky_k.\]
Since $N_1y_1 = 1\pmod{n_1}$, we have $a_1N_1y_1 = a_1\pmod{n_1}$.
Since $n_1\mid N_j$ for $j\ne 1$, all the other terms vanish,
so $x = a_1\pmod{n_1}$. Similarly, $x = a_i\pmod{n_i}$ for all
$i\in\{1,\dots,k\}$.
\br
To see that the solution is unique modulo $N$, suppose $x$ and
$\widetilde x$ are both solutions. Then $x - \widetilde x = 0\pmod{n_i}$.
Multiplying these congrunces together,
we have $x - \widetilde x = 0\pmod N$.
\end{proof}
\section{Wilson's Theorem}
\begin{thm}
If $p$ is prime, then $(p - 1)! = -1\pmod p$.
\end{thm}
\begin{proof}
Note that the only solutions to $x^2 = 1\pmod p$ are 1 and $-1$,
i.e. 1 and $p-1$ are the only equivalence classes
that are their own inverses modulo $p$. Thus
every element from 2 to $p - 2$ has an inverse that isn't itself.
Multiplying the $(p - 3)/2$ classes together gives the result.
\end{proof}
\section{Binomials Modulo $p$}
Note that the binomial coefficients are divisible modulo $p$, for if 
$N = \frac{p!}{(p-r)!r!}$ then $p\mid p!$ but
$p\nmid (p-r)!$ and $p\nmid r!$, thus implying $p\mid N$.
Thus,
\[(a + b)^p = a^p + b^p\pmod p.\]
\chapter{Arithmetic Functions}
An \textit{arithmetic function} is a function from $\ZZ^+$ to $\ZZ$.
\br
One example of an arithmetic function is $(n, k)$ for fixed $k$.
\begin{prop}
For coprime $m$ and $n$,
\[(m, k)(n, k) = (mn, k).\]
\end{prop}
\begin{proof}
We will show $(mn, k)\mid (m, k)(n, k)$ and $(m, k)(n, k)\mid(mn, k)$.
Note that $(m, k)(n, k)$ certainly divides both $mn$ and $k$,
and thus also divides $(mn, k)$. Since we have $(m, k) = am + bk$ and
$(n, k) = cn + dk$ by Bezout's identity,
\[(m, k)(n, k) = mn\cdot ac + (b(cm + dk) + amd)k\]
i.e. $(mn, k)\mid(m, k)(n, k)$. This completes the proof.
\end{proof}
Several other such functions also exist.
\section{The M\"obius Function}
\begin{defn}
The \textit{M\"obius function} is
\[\mu(n) = 
\begin{cases}
1	&\text{if }n=1\\
(-1)^s &\text{if $n$ is squarefree with $s$ prime factors}\\
0 &\text{otherwise}
\end{cases}
\]
\end{defn}
\begin{prop}
For coprime $m$ and $n$,
\[\mu(mn) = \mu(m)\mu(n)\]
i.e. $\mu$ is multiplicative.
\end{prop}
\begin{proof}
By cases. Suppose (without loss of generality)
that $m = 1$. Then $mn = n$, and in particular
\[\mu(mn) = \mu(n) = 1\cdot\mu(n) = \mu(1)\mu(n) = \mu(m)\mu(n).\]
Now suppose $m$ and $n$ are coprime integers both not equal to 1.
If (without loss of generality) $m$ is not squarefree, then
$mn$ will also be not squarefree, wherein
\[\mu(mn) = 0 = 0\cdot\mu(n) = \mu(m)\mu(n).\]
If $m$ and $n$ are both squarefree, then $mn$ will also be 
squarefree. Since $m$ and $n$ are coprime,
$m$ having $s$ divisors and $n$ having $t$ divisors implies $mn$
has $s + t$ divisors.
\end{proof}
\begin{thm}[M\"obius Inversion Formula]
If $f$ and $g$ are such that
\[f(n) = \sum_{d\mid n}g(d),\quad n\in\ZZ^+\]
then equivalently
\[g(n) = \sum_{d\mid n}\mu(d)f(n/d),\quad n\in\ZZ^+.\]
\end{thm}
\begin{proof}
Define the \textit{convolution} of any two arithmetic functions
$f, g$ as
\[(f * g)(n) = \sum_{d\mid n}f(d)g(n/d).\]
Rewriting the sum as
\[(f * g)(n) = \sum_{ab = n}f(a)g(b)\]
makes it clear that convolution is both commutative and associative.
\br
Now we will show that
\[\mu * \boldsymbol 1 = \delta\]
where 
\[\delta(n) = \begin{cases}
1 &\text{if }n=1\\
0 &\text{otherwise}
\end{cases}\]
and $\boldsymbol 1(n) = 1$
for all $n$.
\br
If $n = 1$, then $\sum_{d\mid 1}\mu(n) = \mu(1) = 1$.
So suppose $n\ne 1$ with $k$ prime factors.
All the non-squarefree factors of $n$
vanish in the sum, so
\begin{align*}
\sum_{d\mid n}\mu(d) &= \sum_{\ell= 0}^k{k\choose\ell}
\mu\left(p_1\cdots p_\ell\right)
= (1 - 1)^k = 0.
\end{align*}
Now we prove the formula. Observe that
\[g = \delta * g = (\mu*\boldsymbol 1)*g 
= \mu*(\boldsymbol 1 * g) = \mu * f\]
and
also 
\[f = f *\delta = f*(\mu *\boldsymbol 1) = (f*\mu)*\boldsymbol 1
= g*\boldsymbol 1\]
which is what we wanted to show.
\end{proof}
\begin{prop}
\[\prod_{p\mid n}(1 - p^{-1}) = \sum_{d\mid n}\frac{\mu(d)}{d}.\]
\end{prop}
\begin{proof}
All the non-squarefree factors of $n$ vanish in the sum on the
right, and multiplying out the product on the left yields 
the remaining sum.
\end{proof}
\section{The Euler Totient}
\begin{defn}
The \textit{Euler totient function} is
\[\phi(n) = n\prod_{p\mid n}(1 - p^{-1})\]
where the product is over all primes dividing $n$.
\end{defn}
\begin{prop}
The $\phi$ function counts the integers coprime to $n$:
\[\phi(n) = |\{k: (n, k) = 1, 1\le k < n\}|.\]
\end{prop}
\begin{proof}
When $p\mid n$, the number of positive integers up to $n$
divisible by $p$ is $n/p$.
Thus, each $(1 - p^{-1})$ term in the product filters out the integers
divisible by $p$. For example, if $n = \prod_{j = 1}^kp_j^{a_j}$,
then there are
\[n(1 - p_1^{-1}) = n - n/p_1\]
integers between 1 and $n$ not divisible by $p_1$. Having
a $(1 - p_i^{-1})$ term for each $p_i$ results in the product
counting
the positive integers up to $n$ coprime to $n$.
\end{proof}
\begin{prop}
For coprime $m$ and $n$,
\[\phi(mn) = \phi(m)\phi(n),\]
i.e. $\phi$ is multiplicative.
\end{prop}
\begin{proof}
Consider the system of congruences
\[x = a\pmod m,\qquad x = b\pmod n.\]
Since $m$ and $n$ are coprime, this system has a unique solution
modulo $mn$ by Sun's remainder theorem.
We claim $x$ is coprime to $mn$ if and only if $a$ is coprime to $m$
and $b$ is coprime to $n$.
\br
$(\implies):$
Suppose $x$ is coprime to $mn$.
Then $x$ is coprime to both $m$ and $n$.
Write $x = km + a$ and $x = \ell n + b$.
Were $a$ not coprime to $m$, $x$ would be not coprime to $m$
(since $m$ is not coprime to $m$), so $a$ must be coprime to $m$.
Similarly, $b$ must be coprime to $n$.
\br
$(\impliedby):$
Now suppose $a$ is coprime to
$m$ and $b$ is coprime to $n$.
Again consider $x = km + a$ and $x = \ell n + b$. Were $x$ not
coprime to $m$, then $a$ would not be coprime to $m$, so $x$ must
be coprime to $m$. Similarly, $x$ is coprime to $n$. Since $m$ and $n$
are coprime, $x$ is coprime to $mn$.
\br
Since there are $\phi(m)$ numbers coprime to $m$ and $\phi(n)$ numbers
coprime to $n$, and since
each pair $(a, b)$ produces a unique number $x$
coprime to $mn$, it follows that there are $\phi(m)\phi(n)$
numbers between 1 and $mn$ coprime to $mn$.
\end{proof}
\begin{prop}
\[n = \sum_{d\mid n}\phi(d).\]
\end{prop}
\begin{proof}
We want to show $\id = \boldsymbol 1 *\phi$, so by M\"obius inversion
it suffices to show $\phi = \mu*\id$. From
the definition of $\phi$ and a previous proposition,
\[\phi(n) = n\prod_{p\mid n}(1 - p^{-1})
= \sum_{d\mid n}\mu(d)\frac{n}{d}
= (\mu * \id)(n).\]
This proves the result.
\end{proof}
\begin{prop}
\[\sum_{\ell=1}^n\left\lfloor\frac{n}{\ell}\right\rfloor\phi(\ell)
= {n\choose 2}.\]
\end{prop}
\begin{proof}
Since
\[n = \sum_{d\mid n}\phi(d),\]
we have
\[{n\choose 2} = \sum_{k=1}^nk = \sum_{k=1}^n\sum_{d\mid k}\phi(d)
= \sum_{k=1}^n\sum_{\ell=1}^n\phi(\ell)[\ell\mid k],\]
where 
\[[\ell\mid k] = \begin{cases}
1 &\text{if } \ell\mid k\\
0 &\text{otherwise}
\end{cases}\]
noting that for $\ell > k$ we have $[\ell\mid k] = 0$.
\br
Swapping the order of summation,
\[\sum_{k=1}^n\sum_{\ell=1}^n\phi(\ell)[\ell\mid k]
= 
\sum_{\ell=1}^n\phi(\ell)\sum_{k=1}^n[\ell\mid k]
= \sum_{\ell=1}^n\phi(\ell)\left\lfloor\frac{n}{\ell}\right\rfloor,
\]
which completes the proof.
\end{proof}
\section{Euler's Theorem}
\begin{thm}
If $a$ and $n$ are coprime positive integers, then
\[a^{\phi(n)} = 1\pmod n.\]
\end{thm}
\begin{proof}
For any two integers $a$ and $b$ both coprime to $n$,
their product is also coprime to $n$. Said another way,
\[\prod_{(b, n) = 1}b
= \prod_{(b, n) = 1}ab = a^{\phi(n)}\prod_{(b, n) = 1}b\pmod n,\]
from which the result follows.
\end{proof}
We note that the case $\phi(p) = p - 1$ is known as
Fermat's Little Theorem.
\section{The Sum of Divisors}
\begin{defn}
The \textit{sum of divisors function} is
\[\sigma_k(n) = \sum_{d\mid n}d^k.\]
\end{defn}
\begin{thm}
For coprime $m$ and $n$,
\[\sigma_k(mn) = \sigma_k(m)\sigma_k(n).\]
\end{thm}
\begin{proof}
We'll show that if $f$ and $g$ are multiplicative, then
so is $f * g$. 
\begin{align*}
(f*g)(mn)
&= \sum_{ab = mn}f(a)g(b)\\
&= \sum_{a_mb_m = m}\sum_{a_nb_n = n}f(a_ma_n)g(b_mb_n)\\
&= \sum_{a_mb_m = m}\sum_{a_nb_n = n}f(a_m)f(a_n)g(b_m)g(b_n)\\[0.5em]
&=\left(\sum_{a_mb_m = m}f(a_m)g(b_m)\right)\left(\sum_{a_nb_n = n}
f(a_n)g(b_n)\right)\\[0.5em]
&=
(f*g)(m)\cdot(f*g)(n)
\end{align*}
With this established, note that $\sigma_k = \id^k *\boldsymbol 1$.
This proves the result.
\end{proof}
\chapter{Primitive Roots}
\section{The Order of an Integer}
By Euler's theorem, the set of positive integers $x$
satsifying
\[a^x = 1\pmod n\]
is nonempty.
\begin{defn}
The smallest positive integer $x$ satisfying the above congruence
is denoted $\ord_n(a)$ and is called the \textit{order}
of $a$ modulo $n$.
\end{defn}
\begin{prop}
If $a$ and $n$ are coprime with $n > 0$, then the positive integer
$x$ is a solution to $a^x = 1\pmod n$ if and only if
\[\ord_n(a)\mid x.\]
\end{prop}
\begin{proof}
Suppose $\ord_n(a)\mid x$. Then $x = \ord_n(a)\cdot k$ for some $k$,
hence
\[a^x = a^{\ord_n(a)\cdot k} = (a^{\ord_n(a)})^k = 1^k = 1\pmod n.\]
Conversely, if $a^x = 1\pmod n$, divide to obtain
\[x = q\cdot\ord_n(a) + r,\qquad 0\le r <\ord_n(a).\]
Thus $a^x = a^r\pmod n$. But we must have $r = 0$,
since $y = \ord_n(a)$ is the smallest positive integer such that
$a^y = 1\pmod n$. Hence $\ord_n(a)\mid x$, as desired.
\end{proof}
So, in particular, $\ord_n(a)\mid\phi(n)$.
\begin{prop}
Let $a$, $b$, and $n$ be integers
with $\ord(a)$ and $\ord(b)$ coprime and $n > 0$. Then
\[\ord_n(a)\ord_n(b) = \ord_n(ab).\]
\end{prop}
\begin{proof}
Let $\ord_n(a) = x$, $\ord_n(b) = y$, and $\ord_n(ab) = z$.
Note that $z\mid xy$, since
\[(ab)^{xy} = a^{xy}b^{xy} = (a^x)^y(b^y)^x = 1\pmod n.\]
Since $x$ and $y$ are coprime,
\[(ab)^z = 1 \implies 1 = ((ab)^z)^x 
= (a^x)^zb^{xz} = b^{xz}\implies y\mid xz\implies y\mid z\]
where the third implication follows via Euclid's lemma.
Similarly, $x\mid z$. By coprimality of $x$ and $y$ again,
we have $xy\mid z$. We may thus conclude that $xy = z$.
\end{proof}
\section{Existence of Primitive Roots}
\begin{defn}
If $r$ and $n$ are coprime with $n > 0$ and if
\[\ord_n(r) = \phi(n),\]
then $r$ is called a \textit{primitive root} modulo $n$.
\end{defn}
\begin{thm}
Primitive roots exist modulo a prime.
\end{thm}
\begin{proof}
By Fermat's Little Theorem, the equation
\[X^{p-1} - 1 = 0\]
has $p - 1$ solutions modulo $p$.
For any divisor $d$ of $p - 1$ consider the factorization
\[X^{p-1} - 1 = (X^d - 1)(1 + X^d +\cdots + X^{p - 1 - d}).\]
The polynomial $X^d - 1$ has at most $d$ roots and the other
one has at most $p - 1 - d$ roots and $X^{p-1} - 1$ has
exactly $p - 1$ roots. Hence, $X^d - 1$ has exactly $d$ roots.
\br
Factor $p - 1$ into
\[p - 1 = \prod q^{e_q}\]
For each factor $q^e$ of $p - 1$, $x^{q^e} - 1$ has $q^e$ roots
and $x^{q^{e - 1}} - 1$ has $q^{e - 1}$ roots; hence,
there are $q^e - q^{e - 1} = \phi(q^e)$ elements $x_q$ for which
$\ord_p(x_q) = q^e$.
By the proposition about $\ord_n(a)$ respecting multiplication
with coprime factors,
any product $\prod_q x_q$ has order $p - 1$, and thus is a primitive
root.
\end{proof}
\begin{thm}
Primitive roots exist modulo an odd prime power.
\end{thm}
\begin{proof}
Let $g$ be a primitive root modulo $p$.
By the binomial theorem,
\[(g + p)^{p - 1} = g^{p-1} + (p-1)g^{p-2}p\pmod{p^2},\]
thus $(g + p)^{p-1}\ne g^{p-1}\pmod{p^2}$,
and in particular either $g^{p-1}\ne 1\pmod{p^2}$ or
$(g+p)^{p-1}\ne 1\pmod{p^2}$. Replace $g$ with $g + p$
if necessary to ensure that $g^{p-1}\ne 1\pmod{p^2}$, i.e.
\[g^{p-1} = 1 + k_1p,\quad p\nmid k_1.\]
Again by the binomial theorem,
\[g^{p(p-1)} = (1 + k_1p)^p = 1 + k_2p^2,\quad p\nmid k_2.\]
So $g$ is now a primitive root modulo $p^2$. Let $e > 2$ be
an integer. Again
by the binomial theorem,
\[g^{p^{e-2}(p-1)} = 1 + k_{e-1}p^{e-1},\quad p\nmid k_{e-1}.\]
We have that $\ord_{p^e}(g)\mid\phi(p^e) = p^{e-1}(p-1)$. Note
that $\ord_{p^e}(g)$ can't be of the form $p^{\eps}d$
where $\eps\le e - 1$ and $d$ a proper divisor of $p -1$
because
then \[g^{p^\eps d} = 1\pmod{p^e}\] reduces mod $p$ to $g^d = 1\pmod p$,
contradicting the fact that $g$ is a primitive root modulo $p$.
So we must have \[\ord_{p^e}(g) = p^\eps(p-1)\] where $\eps\le e - 1$,
and the calcluation above shows $\eps = e - 1$, completing
the proof.
\end{proof}
\chapter{Quadratic Residues}
\begin{defn}
If $m$ is a positive integer, we say $a$ is a \textit{quadratic residue}
of $m$ if $(a, m) = 1$ and
\[x^2 = a\pmod m\]
has a solution. If the congruence above has no solution, then
$a$ is a \textit{quadratic nonresidue} of $m$.
\end{defn}
\begin{prop}
Let $p$ be an odd prime and $a$ an integer not divisible by $p$.
Then
\[x^2 = a\pmod p\]
either has no solutions or exactly two distinct (i.e. incongruent)
solutions modulo $p$.
\end{prop}
\begin{proof}
If $x^2 = a \pmod p$ has a solution $x_0$, then $-x_0$ is also a
solution. If $x_0 = -x_0\pmod p$ then $2x_0 = 0\pmod p$, and we may 
divide through by 2 since $p$ is odd, showing that $p\mid x_0$,
contradiction. So there are at least two distinct solutions.
\br
To see that there are exactly two distinct solutions,
suppose $x_0$ and $x_1$ both solve
$x^2 = a\pmod p$. Then $x_0^2 = x_1^2\pmod p$, hence
\[(x_0 - x_1)(x_0 + x_1) = 0\pmod p,\]
implying that $x_0 =\pm x_1$.
\end{proof}
\begin{prop}
If $p$ is an odd prime, there are exactly $\frac{p-1}{2}$ residues
and $\frac{p-1}{2}$ nonresidues of $p$ among the integers
\[1,\quad\dots,\quad p-1.\]
\end{prop}
\begin{proof}
Since each square from $1^2$ to $(p-1)^2$ has exactly
two distinct solutions among 1 through $p-1$,
the conclusion follows.
\end{proof}
\section{The Legendre Symbol}
\begin{defn}
Let $p$ be an odd prime and $a$ an integer. We define
\[\left(\frac{a}{p}\right) = \begin{cases}
1 &\text{if $a$ is a quadratic residue of $p$}\\
-1 &\text{if $a$ is a quadratic nonresidue of $p$}\\
0 &\text{if $a\mid p$}
\end{cases}\]
\end{defn}
\begin{prop}[Euler's criterion]
Let $p$ be an odd prime and $a$ an integer not divisible by $p$.
then
\[\left(\frac{a}{p}\right) = a^\frac{p-1}{2}\pmod p.\]
\end{prop}
\begin{proof}
First assume that 
$\left(\frac{a}{p}\right) = 1$. Then $x^2 = a$ has a solution,
say $x_0$. By Fermat's Little Theorem,
\[a^\frac{p-1}{2} = (x_0^2)^\frac{p-1}{2} = x_0^{p-1} = 1\pmod p.\]
Now assume that 
$\left(\frac{a}{p}\right) = -1$. Then $x^2 = a$ has no solutions,
Note that for each $i$ in 1 through $p-1$ there exists a unique $j$
in 1 through $p-1$ for which $ij = a$, and since $x^2 = a\pmod p$ has
no solutions, we know $i\ne j$. So then
\[(p-1)! = a^\frac{p-1}{2},\]
and applying Wilson's theorem completes the proof.
\end{proof}
\begin{thm}
Let $p$ be an odd prime and $a, b$ integers not divisible by $p$.
Then
\begin{enumerate}
\item if $a = b\pmod p$ then
$\left(\frac{a}{p}\right) = \left(\frac{b}{p}\right)$.
\item 
$\left(\frac{a}{p}\right)\left(\frac{b}{p}\right) = 
\left(\frac{ab}{p}\right)$.
\item
$\left(\frac{a^2}{p}\right) = 1$.
\end{enumerate}
\end{thm}
\begin{proof}
\begin{enumerate}
\item If $a = b\pmod p$, then $x^2 = a\pmod p$
has solutions if and only if $x^2 = b\pmod p$
has solutions, so
$\left(\frac{a}{p}\right) = \left(\frac{b}{p}\right)$.
\item By Euler's criterion,
\[\left(\frac{a}{p}\right)\left(\frac{b}{p}\right) =
a^\frac{p-1}{2}b^\frac{p-1}{2} = (ab)^\frac{p-1}{2} = 
\left(\frac{ab}{p}\right)\pmod p,\]
and since the Legendre symbol takes the values $\pm 1$,
we may conclude that
$\left(\frac{a}{p}\right)\left(\frac{b}{p}\right) = 
\left(\frac{ab}{p}\right)$.
\item This follows from the previous part.
\end{enumerate}
\end{proof}
\begin{prop}
If $p$ is an odd prime, then
\[\left(\frac{-1}{p}\right) =\begin{cases}
~1&\text{if $p =~1~\pmod 4$}\\
-1&\text{if $p = -1\pmod 4$}\\
\end{cases}\] 
\end{prop}
\begin{proof}
Apply Euler's criterion. If $p = 1\pmod 4$, then
\[(-1)^\frac{p- 1}{2} = (-1)^{2k} = 1.\]
If $p = -1\pmod 4$, then
\[(-1)^\frac{p- 1}{2} = (-1)^{2k-1} = -1.\]
\end{proof}
\section{Gauss' Lemma}
\begin{thm}
Let $p$ be an odd prime and $a$ an integer coprime to $p$.
If $s$ is the least number of positive residues modulo $p$
of the integers
\[a,\quad 2a,\quad\dots,\quad\frac{p-1}{2}a\]
that are greater than $p/2$, then
\[\left(\frac{a}{p}\right) = (-1)^s.\]
\end{thm}
\begin{proof}
Let $u_1,\dots,u_s$ represent the residues of the integers
\[a,\quad 2a,\quad\dots,\quad\frac{p-1}{2}a\]
greater than $p/2$, and let $v_1,\dots,v_t$ represent
the residues of these integers less than $p/2$. We will show
\[\{p-u_1,\dots, p-u_s, v_1,\dots,v_t\} = \{1,\dots, p-1\}.\]
It suffices to show that no two of these numbers
are congruent modulo $p$. Were $u_i = u_j$,
then since $a$ does not divide $p$,
\[ma = na\pmod p\implies m = n\pmod p,\]
contradiction. So $u_i\ne u_j$, and similarly $v_i\ne v_j$.
In addition, we cannot have $p - u_i = v_j$, for if so, then
\[ma = p - na\pmod p\implies m= -n\pmod p,\]
which contradicts the fact that $m$ and $n$ are both
in $1$ through $\frac{p-1}{2}$.
\br
Now we multiply things together. We know
\begin{align*}
(p - u_1)\cdots(p - u_s)v_1\cdots v_t
&= (-1)^su_1\cdots u_sv_1\cdots v_t\\
&= (-1)^s\left(\frac{p-1}{2}\right)!\pmod p
\end{align*}
Yet at the same time,
\begin{align*}
u_1\cdots u_sv_1\cdots v_t = a^\frac{p-1}{2}\left(\frac{p-1}{2}\right)!\pmod p
\end{align*}
By Euler's criterion,
\[\left(\frac{a}{p}\right) = a^\frac{p-1}{2} = (-1)^s,\]
which completes the proof.
\end{proof}
\begin{prop}
If $p$ is an odd prime, then
\[\left(\frac{2}{p}\right) = (-1)^\frac{p^2 - 1}{8}.\]
\end{prop}
\begin{proof}
First, we compute the number of residues in
\[1\cdot 2,\quad 2\cdot 2,\quad\cdots,\quad \frac{p-1}{2}\cdot 2\]
greater than $p/2$. This is a direct count since all of the
above residues are less than $p$. When $1\le j\le\frac{p-1}{2}$,
$2j < p/2$ when $j\le p/4$, so there are $\lfloor\frac{p}{4}\rfloor$
integers less than $p/2$, and thus
\[s = \frac{p-1}{2} - \left\lfloor\frac{p}{4}\right\rfloor\]
greater than $p/2$.
By Gauss' lemma, it remains to show that
\[\frac{p^2 - 1}{8} = \frac{p-1}{2}
- \left\lfloor\frac{p}{4}\right\rfloor\pmod 2.\]
We first consider $\frac{p^2 - 1}{8}$. If $p = \pm 1\pmod 8$,
then
\[\frac{p^2 - 1}{8} = \frac{64k^2\pm 16k}{8} = 0\pmod 2.\]
If $p = \pm 3\pmod 8$, then
\[\frac{p^2 - 1}{8} =\frac{64k^2\pm 48k + 8}{8} =  1\pmod 2.\]
Now we consider
$x = \frac{p-1}{2} - \left\lfloor\frac{p}{4}\right\rfloor$.
\begin{align*}
p = 8k + 1 &\implies x = 4k -
\left\lfloor 2k + \frac{1}{4}\right\rfloor = 0\pmod 2\\[1em]
p = 8k + 3 &\implies x = 4k + 1 -
\left\lfloor 2k + \frac{3}{4}\right\rfloor = 1\pmod 2\\[1em]
p = 8k + 5 &\implies x = 4k + 2 -
\left\lfloor 2k + \frac{5}{4}\right\rfloor = 1\pmod 2\\[1em]
p = 8k + 7 &\implies x = 4k + 3 -
\left\lfloor 2k + \frac{7}{4}\right\rfloor = 0\pmod 2
\end{align*}
Since $\frac{p^2 - 1}{8} = \frac{p-1}{2}
- \left\lfloor\frac{p}{4}\right\rfloor\pmod 2$
in all cases, the proof is complete.
\end{proof}
\section{The Law of Quadratic Reciprocity}
\begin{prop}
If $p$ is an odd prime and $a$ an integer not divisible by $p$,
then
\[\left(\frac{a}{p}\right) = (-1)^{T(a, p)}\]
where
\[T(a, p) = \sum_{j=1}^\frac{p-1}{2}
\left\lfloor\frac{aj}{p}\right\rfloor\]
\end{prop}
\begin{proof}
As in the proof of Gauss' lemma, let $u_1,\dots, u_s$
represent the residues of
\[a,\quad 2a,\quad\dots,\quad\frac{p-1}{2}a\]
that are greater than $p/2$, and $v_1,\dots,v_t$ the
residues of the above numbers that are less than $p/2$.
Dividing,
\[ja = p\left\lfloor\frac{aj}{p}\right\rfloor + r\]
where $r = u_i$ or $r = v_j$.
Adding $\frac{p-1}{2}$ of these together yields
\[\sum_{j=1}^\frac{p-1}{2}ja
= \sum_{j=1}^\frac{p-1}{2}p\left\lfloor\frac{aj}{p}\right\rfloor
+ \sum_{i=1}^su_i + \sum_{j=1}^tv_j \]
We also showed, though, that $p - u_1,\dots p - u_s, v_1,\dots, v_t$
are all the integers from 1 through $\frac{p-1}{2}$, so
\[\sum_{j=1}^\frac{p-1}{2}j = ps
- \sum_{i=1}^su_i + \sum_{j=1}^tv_j.\]
Subtracting these equations, we find
\[(a - 1)\sum_{j=1}^\frac{p-1}{2}j
= pT(a, p) - ps + 2\sum_{i=1}^su_i\]
and since $a$ and $p$ are odd, this reduces mod 2 to
\[T(a, p) = s\pmod 2,\]
and applying Gauss' lemma completes the proof.
\end{proof}
\begin{thm}[Quadratic Reciprocity]
Let $p$ and $q$ be odd primes. Then
\[\left(\frac{p}{q}\right)\left(\frac{q}{p}\right)
= (-1)^{\frac{p-1}{2}\frac{q-1}{2}}.\]
\end{thm}
\begin{proof}
We consider pairs of integers $(x, y)$ where $1\le x\le\frac{p-1}{2}$
and $1\le y\le \frac{q-1}{2}$. There are $\frac{p-1}{2}\frac{q-1}{2}$
such pairs. We divide these pairs into two groups based on relative
sizes of $qx$ and $py$.
\br
First we note that for all such pairs $(x, y)$ we have $qx\ne py$, for
if $qx = py$, then $q\mid py$, implying either $q\mid p$ or $q\mid y$.
But $q\mid p$ cannot happen since $q$ and $p$ are distinct primes,
and $q\mid y$ cannot happen since $1\le y\le \frac{q-1}{2}$.
\br
To count the pairs for which $qx >  py$,
note that these are the pairs for
which $1\le x\le \frac{p-1}{2}$ and $1\le y\le\frac{qx}{p}$, hence
their number is $T(q, p)$.
\br
To count the pairs for which $qx <  py$,
note that these are the pairs for
which $1\le y\le \frac{q-1}{2}$ and $1\le x\le\frac{py}{q}$, hence
their number is $T(p, q)$.
\br
So 
\[T(q, p) + T(p, q) = \frac{p-1}{2}\frac{q-1}{2},\]
hence
\[\left(\frac{p}{q}\right)\left(\frac{q}{p}\right)
= (-1)^{T(q, p) + T(p, q)}
= (-1)^{\frac{p-1}{2}\frac{q-1}{2}},\]
as desired.
\end{proof}
\end{document}
