%
% finite-dim.tex
% -- komplexchemie, 18.vii.25
%
\documentclass[12pt]{memoir}
\usepackage[margin=1cm]{geometry}
\usepackage{palatino, euler, mathtools, nopageno, amsthm, amssymb}
\def\br{~\\[1em]}
\def\inv{^{-1}}
\DeclareMathOperator{\Span}{Span}
\begin{document}
~\\[0em]
\begin{center}
\Large Finite $\implies$ (Injective $\iff$ Surjective)
\end{center}
~\\[-2em]
\begin{abstract}
\noindent 
This writeup establishes two facts:
\begin{enumerate}
\item If $f:S\to S$ where $|S| < \infty$,
then $f$ injects iff $f$ surjects.
\item If $T:V\to V$ where $\dim V < \infty$,
then $T$ injects iff $T$ surjects.
\end{enumerate}
These are basically the same thing; we hope to make this
clear by juxtaposing the arguments.
\end{abstract}
\subsection*{Finite Sets:\quad Injective $\implies$ Surjective}
Let $S$ be a nonempty finite set with $N$ elements and let $f:S\to S$.
\begin{center}
\textit{\textbf{Claim:} If $f$ is injective, then $f$ is surjective.}
\end{center}
\begin{proof}
Suppose $f$ is injective.
Let $s\in S$, and consider the elements
\[s,\quad f(s),\quad f^2(s),\quad\dots,\quad f^N(s)\]
in $S$. Since $S$ only has $N$ elements, we have
\[f^m(s) = f^M(s)\]
for some $0\le m\le M$ (without loss of generality).
But then by injectivity of $f$, we have
\[s = f(f^{M - m - 1}(s)),\]
which shows that $f$ is surjective.
\end{proof}
~\\[-2.5em]
\subsection*{Finite Sets:\quad Surjective $\implies$ Injective}
Let $S$ be a nonempty finite set with $N$ elements and let $f:S\to S$.
\begin{center}
\textit{\textbf{Claim:} If $f$ is surjective, then $f$ is injective.}
\end{center}
\begin{proof}
Suppose $f$ is surjective. Pick $a, b\in S$ and suppose
\[f(a) = f(b) =: c.\] Then $S\setminus\{c\}$ has $N - 1$ elements,
hence $f\inv(S\setminus\{c\}) = S\setminus f\inv(\{c\})$ has at least
$N - 1$ elements, hence $f\inv(\{c\})$ has at most one element,
hence $a = b$, which shows that $f$ is injective.
\end{proof}
~\\[-2.5em]
\subsection*{Finite Sets with Zero Elements}
If $S = \varnothing$, then the only map from $S$ to $S$ is the null map,
which consists of zero ordered pairs and is vacuously bijective.
\vfill\pagebreak
\subsection*{Finite Dimensional Vector Spaces:
\quad Injective $\implies$ Surjective}
Let $V$ be a nontrivial vector space of dimension $N$,
and let $T:V\to V$.
\begin{center}
\textit{\textbf{Claim:} If $f$ is injective, then $f$ is surjective.}
\end{center}
\begin{proof}
Suppose $T$ is injective, i.e. suppose $\ker T = \{0\}$. Let $v\in V$.
\br
Consider the vectors:
\[v,\quad Tv,\quad T^2v,\quad\dots,\quad T^Nv.\]
Since $V$ has dimension $N$, this is a linearly dependent set,
i.e. there exists $\lambda = (\lambda_0,\dots,\lambda_N)\ne 0$
such that
\[\lambda_0v + \lambda_1Tv + \cdots + \lambda_NT^Nv = 0.\]
Suppose $\lambda_0\ne 0$. Then we may divide by $\lambda_0$
and use linearity of $T$ to obtain:
\[v = T\left(-\frac{1}{\lambda_0}
(\lambda_1v + \cdots + \lambda^NT^{N-1}v)\right).\]
If $\lambda_0 = 0$, then $\lambda_1Tv + \cdots + \lambda_NT^Nv = 0$,
and by linearity of $T$, we get
$T(\lambda_1v + \cdots + \lambda_NT^{N-1}v) = 0$, i.e.
\[\lambda_1v + \cdots + \lambda_NT^{N-1}v\in\ker T.\]
But since $T$ is injective, this implies
\[\lambda_1v + \cdots + \lambda_NT^{N-1}v = 0,\]
and so we're back where we started. Since $\lambda\ne 0$,
we eventually encounter a nonzero $\lambda_i$, which we may divide by,
and then use linearity of $T$ as before.
This proves $T$ is surjective.
\end{proof}
~\\[-2em]
\subsection*{Finite Dimensional Vector Spaces:\quad
Surjective $\implies$ Injective}
Let $V$ be a nontrivial vector space of dimension $N$,
and let $T:V\to V$.
\begin{center}
\textit{\textbf{Claim:} If $f$ is surjective, then $f$ is injective.}
\end{center}
\begin{proof}
Suppose $T$ is surjective, i.e. suppose $T(V) = V$.
Let $v\in\ker T$. Then $\Span(v)\subset\ker T$, so there
exists a unique surjective linear map
\[\widetilde T:V/\Span(v)\to V,\]
which implies $\dim(V/\Span(v))\ge N$. But we also have
\[\dim(V/\Span(v)) = N - \dim(\Span(v))\le N,\]
so $V/\Span(v)$ has dimension $N$.
This implies $\Span(v) = \{0\}$, hence $v = 0$.
This proves $T$ is injective.
\end{proof}
~\\[-2em]
\subsection*{Vector Spaces of Dimension Zero}
If $V = \{0\}$, the only map from $V$ to $V$ is the zero map, which
sends 0 to 0 and is trivially bijective.
\end{document}
