%
% integration.tex
% -- komplexchemie, 18.vii.25
%
\documentclass[12pt]{memoir}
\usepackage[margin=1cm]{geometry}
\usepackage{palatino, euler, mathtools, nopageno}
\def\br{~\\[1em]}
\def\dd{\mathrm d}
\def\ee{\mathrm e}
\def\ii{\mathrm i}
\def\RR{\mathbf R}
\begin{document}
~\\[0em]
\begin{center}
\Large Single Variable Calculus: Integration Basics 
\end{center}
~\\[-2em]
\begin{abstract}
This is a simple tutorial on how to generate antiderivatives.
We emphasize using the differential $\dd x$ as a function,
in order to make the classic techniques (substitution,
integration by parts) more intuitive. Instead of drilling
arduous trigonometric substitutions, we work with Euler's formula.
The general theme is to emphasize \textbf{pattern recognition}
over rote memorization.
\br
Though not absolutely required,
some familiarity with implicit differentiation may be helpful.
\end{abstract}
~\\[0em]
\textit{In this writeup, $\log x$ denotes the natural logarithm.}
\br
The key to understanding the basic integration techniques is to
have a clear idea of where they come from. Recall the chain rule
and the product rule:
\begin{align*}
\frac{\dd(g\circ f)}{\dd x}(x)
&= \frac{\dd g}{\dd x}(f(x))\cdot\frac{\dd f}{\dd x}(x)\\[1em]
\frac{\dd(fg)}{\dd x}(x)
&= \frac{\dd f}{\dd x}(x)\cdot g(x)
+ f(x)\cdot\frac{\dd g}{\dd x}(x)
\end{align*}
\textbf{Integration via substitution is a matter of recognizing
the chain rule in reverse.} For example,
\[\dd(x^2) = 2x\,\dd x\implies\int 2x\,\dd x
= \int\dd(x^2) = x^2 + C.\]
Here are a few more:
\begin{align*}
\int\frac{\sin x\,\dd x}{1 + \cos^2 x}
&= -\int\frac{d(\cos x)}{1 + \cos^2 x}
= -\int\frac{du}{1 + u^2}
= -\arctan u + C_0
= \boxed{-\arctan(\cos x) + C}\\[1.5em]
\int x^2\ee^{x^3}\,dx &= \frac{1}{3}\int\ee^{x^3}\,\dd(x^3)
= \frac{1}{3}\int\ee^u\,\dd u
= \frac{1}{3}\ee^u + C_0
= \boxed{\frac{1}{3}\ee^{x^3} + C}\\[1.5em]
\int\frac{\dd x}{x\log x}
&= \int\frac{1}{\log x}\dd(\log x)
= \int\frac{\dd u}{u}
= \log u + C_0
= \boxed{\log\log x + C}\\[1.5em]
\int\frac{x^3\,\dd x}{\sqrt{1 + x^2}}
&= \frac{1}{2}\int\frac{x^2\,\dd(x^2)}{1 + x^2}
= \frac{1}{2}\int\frac{u\,\dd u}{\sqrt{1 + u}}
= \frac{1}{2}\int\frac{y^2 - 1}{y}2y\,\dd y
= \int y^2 - 1\,\dd y\\[0.5em]
&= \frac{1}{3}y(y^2 - 3) + C_0
= \frac{1}{3}\sqrt{1 + u}(u - 2) + C_1
= \boxed{\frac{1}{3}\sqrt{1 + x^2}(x^2 - 2) + C}
\end{align*}
From this viewpoint, the game is to find instances of 
$\dd(f(x)) = f'(x)\,\dd x$.
\vfill\pagebreak
\noindent
Similarly, 
\textbf{integration by parts is a matter of recognizing the product
rule in reverse.} For example,
\begin{align*}
\int x^2\sin x\,\dd x
&= -\int x^2\,\dd(\cos x)
= -\int\left(\dd\left(x^2\cos x\right) - \cos x\,\dd(x^2)\right)\\
&= C_0 - x^2\cos x + 2\int x\cos x\,\dd x
= C_0 - x^2\cos x + 2\int x\,\dd(\sin x)\\
&= C_0 - x^2\cos x + 2\int\left(\dd(x\sin x) - \sin x\,\dd x\right)\\
&= C_1 - x^2\cos x + 2x\sin x - 2\int\sin x\,\dd x\\
&= \boxed{2x\sin x - (x^2 - 2)\cos x + C}
\end{align*}
From this viewpoint, the game is to find instances of
$f(\dd g) = \dd(fg) - g(\dd f)$.
\br
Finally,
\textbf{instead of trigonometric substutions, recall Euler's formula:}
\[\ee^{\ii\theta} = \cos\theta + \ii\sin\theta\]
Since cosine is even and sine is odd, we also get:
\[\ee^{-\ii\theta} = \cos\theta - \ii\sin\theta\]
After a bit of rearranging we get the following helpful identities:
\[\cos\theta
= \frac{\ee^{\ii\theta} + \ee^{-\ii\theta}}{2},\qquad
\sin\theta
= \frac{\ee^{\ii\theta} - \ee^{-\ii\theta}}{2i}.
\]
Slightly more generally,
\[\cos(n\theta)
= \frac{\ee^{\ii n\theta} + \ee^{-\ii n\theta}}{2},\qquad
\sin(n\theta)
= \frac{\ee^{\ii n\theta} - \ee^{-\ii n\theta}}{2i}.
\]
Here is the technique in action (some algebra steps skipped):
\begin{align*}
\int\sin^3\theta\cos^2\theta\,\dd\theta
&= -\frac{1}{32i}\int
\left(\ee^{3\ii\theta} - \ee^{-3\ii\theta}
- 3\left(\ee^{\ii\theta} - \ee^{-\ii\theta}\right)\right)
\left(\ee^{2\ii\theta} + 2 +
\ee^{-2\ii\theta}\right)\,\dd\theta\\[0.5em]
&= -\frac{1}{32i}\int\left(
\ee^{5\ii\theta} - \ee^{-5\ii\theta}
- \left(
\ee^{3\ii\theta} - \ee^{-3\ii\theta}
\right)
- 2\left(
\ee^{\ii\theta} - \ee^{-\ii\theta}
\right)
\right)\,\dd\theta\\[0.5em]
&= \frac{1}{16}\int\left(-\sin(5\theta)
+ \sin(3\theta) + 2\sin\theta\right)\dd\theta\\[0.5em]
&= \boxed{\frac{1}{80}\cos(5\theta)
- \frac{1}{48}\cos(3\theta) - \frac{1}{8}\cos\theta + C}
\end{align*}
If you're skilled at algebra but not so much at trigonometry,
Euler's formula is your friend.
\br
\hrule
\br
It is sometimes remarked that differentiation is a skill,
whereas integration is an art. Hopefully this writeup communicates
that, in certain contexts,
\textbf{art reduces to applying skill in a thoughtful way}.
\end{document}
