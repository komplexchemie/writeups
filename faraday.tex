%
% faraday.tex
% -- komplexchemie, 18.vii.25
%
\documentclass[12pt]{memoir}
\usepackage[margin=1cm]{geometry}
\usepackage{palatino, euler, mathtools, nopageno, multicol, changepage}
\def\br{~\\[1em]}
\def\EE{\mathbf E}
\def\BB{\mathbf B}
\def\JJ{\mathbf J}
\def\AA{\mathbf A}
\def\dd{\mathrm d}
\def\pd{\partial}
\def\ph{\varphi}
\def\eps{\varepsilon}
\newcommand\dx[1]{\dd x^{#1}}
\newcommand\ddx[2]{\dd x^{#1}\wedge\dd x^{#2}}
\newcommand\dddx[3]{\dd x^{#1}\wedge\dd x^{#2}\wedge\dd x^{#3}}
\begin{document}
~\\[-2em]
\begin{center}
\Large Electrodynamics: Connection and Curvature
\end{center}
~\\[-2em]
\begin{abstract}
\noindent
We derive the differential form of Maxwell's equations carefully,
as well as the matrix form of $F$.
\end{abstract}
~\\[0em]
This writeup uses the $(-, +, +, +)$ signature convention.
Let $c = \mu_0 = \eps_0 = 1$.
\br
Maxwell's equations in vector form are then:
\vspace*{-2em}
\begin{adjustwidth}{5cm}{5cm}
\begin{multicols}{2}
\begin{align*}
\nabla\cdot\EE &= 4\pi\rho\\
\nabla\cdot\BB &= 0
\end{align*}
\vfill
\columnbreak
\begin{align*}
\nabla\times\EE &= -\pd_t\BB\\
\nabla\times\BB &= 4\pi\,\JJ + \pd_t\EE
\end{align*}
\end{multicols}
\end{adjustwidth}
Where $\EE$ is the electric field, $\BB$ is the magnetic field,
and $\JJ$ the electric current.
\br
In tensor notation, we have:
\vspace*{-2em}
\begin{adjustwidth}{5cm}{5cm}
\begin{multicols}{2}
\begin{align*}
\pd_iE^i &= 4\pi\,J^0\\
\pd_iB^i &= 0
\end{align*}
\vfill
\columnbreak
\begin{align*}
\eps^{ijk}\pd_jE_k &= -\pd_0B^i\\
\eps^{ijk}\pd_jB_k &= 4\pi\,J^i + \pd_0E^i
\end{align*}
\end{multicols}
\end{adjustwidth}
The \textbf{electromagnetic 4--potential} $A$
can be expressed as a vector:
\[A^\sharp = \phi e_0 + A^1e_1 + A^2e_2 + A^3e_3
\quad\longleftrightarrow\quad
A^\alpha = \phi e_0 + A^ie_i\]
We may flatten this to obtain a 1-form:
\[A^\flat =
\phi\,\dx{0}
+ A_1\,\dx{1}
+ A_2\,\dx{2}
+ A_3\,\dx{3}
\quad\longleftrightarrow\quad
A_\alpha = \phi\,\dx{0} + A_i\dx{i}
\]
Taking the exterior derivative of this expression,
\begin{align*}
\dd A^\flat
&= \dd\!\left(
\phi\,\dx{0}
+ A_1\,\dx{1}
+ A_2\,\dx{2}
+ A_3\,\dx{3}
\right)\\
&=
\dd\!\left(\phi\,\dx{0}\right)
+ \dd\!\left(A_1\,\dx{1}\right)
+ \dd\!\left(A_2\,\dx{2}\right)
+ \dd\!\left(A_3\,\dx{3}\right)\\[0.5em]
&=
\pd_0\phi\,\ddx{0}{0}
+ \pd_1\phi\,\ddx{1}{0}
+ \pd_2\phi\,\ddx{2}{0}
+ \pd_3\phi\,\ddx{3}{0}\\
&+\pd_0A_1\,\ddx{0}{1}
+ \pd_1A_1\,\ddx{1}{1}
+ \pd_2A_1\,\ddx{2}{1}
+ \pd_3A_1\,\ddx{3}{1}\\
&+\pd_0A_2\,\ddx{0}{2}
+ \pd_1A_2\,\ddx{1}{2}
+ \pd_2A_2\,\ddx{2}{2}
+ \pd_3A_2\,\ddx{3}{2}\\
&+\pd_0A_3\,\ddx{0}{3}
+ \pd_1A_3\,\ddx{1}{3}
+ \pd_2A_3\,\ddx{2}{3}
+ \pd_3A_3\,\ddx{3}{3}\\[0.5em]
&= \left(\pd_0A_1 - \pd_1\phi\right)\,\ddx{0}{1}
+ \left(\pd_0A_2 - \pd_2\phi\right)\,\ddx{0}{2}
+ \left(\pd_0A_3 - \pd_3\phi\right)\,\ddx{0}{3}\\
&+ \left(\pd_2A_3 - \pd_3A_2\right)\,\ddx{2}{3}
+ \left(\pd_3A_1 - \pd_1A_3\right)\,\ddx{3}{1}
+ \left(\pd_1A_2 - \pd_2A_1\right)\,\ddx{1}{2}\\[0.5em]
&= \left(\pd_\mu A_\nu - \pd_\nu A_\mu\right)\,\ddx{\mu}{\nu}
\end{align*}
Expressing the (antisymmetric)
\textbf{electromagnetic tensor / Faraday 2--form} $F$ as
\[F^{\flat\flat} = (\pd_\mu A_\nu - \pd_\nu A_\mu)\,\ddx{\mu}{\nu}
\quad\longleftrightarrow\quad
F_{\mu\nu} = (\pd_\mu A_\nu - \pd_\nu A_\mu)\,\dx{\mu}\otimes\dx{\nu}
\]
we see $\dd A^\flat = F^{\flat\flat}$ and hence $\dd F^{\flat\flat}
= \dd(\dd A^\flat) = 0$. This is commonly simplified to
\[\boxed{\dd F = 0.}\]
\vfill\pagebreak
\noindent
The fields $\EE$ and $\BB$ can be expressed in terms of the potentials
$\phi$ (electric) and $\AA$ (magnetic) as:
\[\EE = -\nabla\phi - \pd_t\AA,\qquad \BB = \nabla\times\AA\]
In tensor notation, this is:
\[E^i = -\pd^iA^0 - \pd_0A^i,\qquad B^i = \eps^{ijk}\pd_jA_k\]
Simplifying,
\begin{align*}
E^i &= -\pd^iA^0 - \pd_0A^i = -\eta^{ii}\pd_iA^0 - \pd_0A^i\\
&= -\pd_iA^0 - \pd_0A^i = -\pd_iA_0\eta^{00} - \pd_0A_i\eta^{ii}\\
&= \pd_iA_0 - \pd_0A_i = -F_{i0}
\end{align*}
\begin{align*}
B^i &= \eps^{ijk}\pd_jA_k = \pd_jA_k - \pd_kA_j = F_{jk}
\end{align*}
so that in matrix form (using antisymmetry of $F$),
\[[F_{\mu\nu}] = \left[
\begin{array}{cccc}
0 & -E^1 & -E^2 & -E^3\\
E_1 & 0 & B^3 & -B^2\\
E_2 & -B^3 & 0 & B^1\\
E_3 & B^2 & -B^1 & 0
\end{array}
\right].\]
Let's rewrite $F$ as a 2--form with these new expressions:
\begin{align*}
F^{\flat\flat}
&= E^1\,\ddx{1}{0}
+ E^2\,\ddx{2}{0}
+ E^3\,\ddx{3}{0}\\
&+ B^1\,\ddx{2}{3}
+ B^2\,\ddx{3}{1}
+ B^3\,\ddx{1}{2}
\end{align*}
Taking the Hodge star, differentiating, and then taking the Hodge star
once more,
\begin{align*}
\star F^{\flat\flat}
&= \star(E^1\,\ddx{1}{0}
+ E^2\,\ddx{2}{0}
+ E^3\,\ddx{3}{0})\\
&+ \star(B^1\,\ddx{2}{3}
+ B^2\,\ddx{3}{1}
+ B^3\,\ddx{1}{2})\\[0.5em]
&= E^1\,\ddx{2}{3}
+ E^2\,\ddx{3}{1}
+ E^3\,\ddx{1}{2}\\
&- B^1\,\ddx{1}{0}
- B^2\,\ddx{2}{0}
- B^3\,\ddx{3}{0}\\[1em]
\dd(\star F^{\flat\flat})
&=
\pd_0E^1\dddx{0}{2}{3} + \pd_1E^1\dddx{1}{2}{3}
+\pd_0E^2\dddx{0}{3}{1} + \pd_2E^2\dddx{2}{3}{1}\\
&+\pd_0E^2\dddx{0}{1}{2} + \pd_3E^3\dddx{3}{1}{2}
- \pd_2B^1\dddx{2}{1}{0} - \pd_3B^1\dddx{3}{1}{0}\\
&- \pd_3B^2\dddx{3}{2}{0} - \pd_1B^2\dddx{1}{2}{0}
- \pd_1B^3\dddx{1}{3}{0} - \pd_2B^3\dddx{2}{3}{0}\\[0.5em]
&= (\pd_1E^1 + \pd_2E^2 + \pd_3E^3)\,\dddx{1}{2}{3}
+ (\pd_0E^1 + \pd_3B^2 - \pd_2B^3)\,\dddx{0}{2}{3}\\
&+ (\pd_0E^2 + \pd_1B^3 - \pd_3B^1)\,\dddx{0}{3}{1}
+ (\pd_0E^3 + \pd_2B^1 - \pd_1B^2)\,\dddx{0}{1}{2}\\[1em]
\star\dd(\star F^{\flat\flat})
&= -(\pd_1E^1 + \pd_2E^2 + \pd_3E^3)\,\dx{0}
+ (\pd_0E^1 + \pd_3B^2 - \pd_2B^3)\,\dx{1}\\
&+ (\pd_0E^2 + \pd_1B^3 - \pd_3B^1)\,\dx{2}
+ (\pd_0E^3 + \pd_2B^1 - \pd_1B^2)\,\dx{3}\\
&= -\pd_iE^i\,\dx{0} + (\pd_0E^i - \eps^{ijk}\pd_jB^k)\,\dx{i}
= -\pd_iE^i\,\dx{0} + (\pd_0E^i - \eps^{ijk}\pd_jB_k\eta^{kk})\,\dx{i}\\
&= -\pd_iE^i\,\dx{0} + (\pd_0E^i - \eps^{ijk}\pd_jB_k)\,\dx{i}
= -4\pi J^0\,\dx{0} + 4\pi J^i\,\dx{i}
= -4\pi J_0\eta^{00}\dx{0} + 4\pi J_i\eta^{ii}\,\dx{i}\\
&= 4\pi J_0\,\dx{0} + 4\pi J_i\,\dx{i}
= 4\pi J_\alpha\dx{\alpha} = 4\pi J^\flat.
\end{align*}
This is commonly simpified to $\star\dd(\star F) = 4\pi J$ or,
using the codifferential $\delta(-) = \star\dd(\star(-))$,
\[\boxed{\delta F = 4\pi J.}\]

\end{document}
