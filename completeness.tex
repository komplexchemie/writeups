%
% completeness.tex
% -- komplexchemie, 24.viii.25
%
\documentclass[oneside, 12pt]{memoir}
\usepackage[margin=1cm, includefoot]{geometry}
\usepackage{charter, euler}
\usepackage{amsthm, shadethm, xcolor, amsmath, microtype}
\newtheoremstyle{elegant}{0.8\baselineskip}{0.1\baselineskip}{\itshape}{}{\scshape}{.}{ }{}
\newtheoremstyle{elegant-defn}{0.8\baselineskip}{0.1\baselineskip}{\normalfont}{}{\scshape}{.}{ }{}
\theoremstyle{elegant}

\definecolor{shadethmcolor}{HTML}{f7f8f8}
\setlength\shadeboxsep{0.8em}
\setlength\shadedtextwidth
{\dimexpr\textwidth-2\shadeboxsep\relax}

\theoremstyle{elegant}
\newshadetheorem{thm}{Theorem}
\def\RR{\mathbf R}
\def\NN{\mathbf N}
\begin{document}
\pagestyle{plain}
~\\[0em]
\begin{center}
\Large Forms of Completeness
\end{center}
~\\[-2em]
\begin{abstract}
We show that the Archimedean property implies the logical equivalence
of six different forms of the completeness of $\RR$: the convergence
of Cauchy sequences, the existence of infima and suprema, that property
about nested closed intervals, the monotone convergence theorem,
the Bolzano-Weierstrass theorem, and the intermediate value theorem.
\end{abstract}
~\\[-2em]
\begin{enumerate}
\setcounter{enumi}{-1}
\item \textbf{Archimedean Property:} Let $\alpha, \beta\in\RR$.
Then $\exists N\in\NN$ such that $\alpha < N\beta$.
\item \textbf{Cauchy Completeness:} Let $(x_i)_i\subset\RR$.
If $(x_i)_i$ is Cauchy, then $x_i\to x_\infty$ for some $x_\infty\in\RR$.
\item \textbf{inf/sup Completeness:} Let $S\subseteq\RR$ be nonempty.\\
If $S$ is bounded above, then $\sup S$ exists. If $S$ is bounded
below, then $\inf S$ exists.
\item \textbf{Interval Refinement Property:} Let
$([x_i - \epsilon_i, x_i + \epsilon_i])_i$
be a sequence of closed intervals such that
\[[x_0 - \epsilon_0, x_0 + \epsilon_0]\supset[x_1 - \epsilon_1, x_1 + \epsilon_1]\supset [x_2 - \epsilon_2, x_2 + \epsilon_2]\supset\cdots
\]
with $\epsilon_i\to 0$.
Then the intersection
$\bigcap_i[x_i - \epsilon_i, x_i + \epsilon_i]$ is nonempty.
\item \textbf{Monotone Convergence Theorem:} Let $(x_i)_i\subset\RR$.
If $(x_i)_i$ is either strictly increasing and bounded above
or strictly decreasing and bounded below, then $x_i\to x_\infty$ for some
$x_\infty\in\RR$.
\item \textbf{Bolzano-Weierstrass Theorem:}
Let $(x_i)_i\subset\RR$. If $(x_i)_i$ is bounded,
then $\exists (x_{\sigma(i)})_i\subseteq (x_i)_i$ such that
$x_{\sigma(i)}\to x_\infty$ for some $x_\infty\in\RR$.
\item \textbf{Intermediate Value Theorem:}
Let $a, b\in \RR$ with $a < b$ and let $f:[a, b]\to\RR$ be continuous.
If $f(a) < 0 < f(b)$, then $f(c) = 0$ for some $c\in (a, b)$.
\end{enumerate}
\hrule
\begin{thm}
Suppose 0 holds. Then 1 through 6 are logically equivalent.
\end{thm}
\begin{proof} We'll do $1\le 2\le 3\le 1$ and then
$2\le 4\le 5\le 6\le 2$.
\begin{itemize}
\item[$1\le 2$] Suppose $S\subset\RR$ is nonempty and bounded above.
Let $s_0\in S$ and let $M_0$ be an upper bound of $S$.
Construct $(s_i)_i$ and $(M_i)_i$ as follows: if $\frac{s_i + M_i}{2}$
is an upper bound of $S$, set $(s_{i+1}, M_{i+1}) = (s_i, \frac{s_i + M_i}{2})$. Otherwise there is some $x\in S$ such that $x\ge\frac{s_i + M_i}{2}$, and in this case we set $(s_{i+1}, M_{i+1}) = (x, M_i)$.
The sequence $(s_0, M_0, s_1, M_1, s_2, M_2,\dots)$ is Cauchy, since
$|s_{i+1} - M_{i+1}|\le\frac{1}{2}|s_i - M_i|$. Let $\alpha$ be the limit
of this sequence. Then $M_i\to\alpha$ and $s_i\to\alpha$.
So $\alpha$ is an upper bound of $S$, for if there were some $\alpha'$ such that $\alpha < \alpha' < M_i$
for all $i\in\NN$, then $\alpha' - \alpha$ would be an $\epsilon$ such that
$|\alpha - M_i|> \epsilon$ for all $i\in\NN$, which cannot happen since $M_i\to\alpha$.
Furthermore, since $s_i\to\alpha$, for every $\epsilon > 0$ there is
some $i\in\NN$ such that 
$\alpha - s_i < \epsilon$, which is precisely the supremum condition.
\item[$2\le 3$] Suppose we have a sequence of nested closed intervals
$([x_i - \epsilon_i, x_i + \epsilon_i])_i$ with $\epsilon_i\to 0$. The set $S = \overline{(x_i - \epsilon_i)_i}$ is nonempty and bounded above by $x_i + \epsilon_i$ for all $i\in\NN$,
and thus has a supremum. Let $\alpha = \sup S$. Then since $\alpha$ is
an upper bound of $S$, $x_i - \epsilon_i\le \alpha$ for all $i\in\NN$.
But since $\alpha$ is the
least upper bound, any upper bound will be at least $\alpha$,
in particular $\alpha\le x_i + \epsilon_i$ for all $i\in\NN$.
Combining these two facts, we see that $\alpha\in[x_i -\epsilon_i, x_i + \epsilon_i]$
for all $i\in\NN$, i.e. the intersection $\bigcap_i[x_i -\epsilon_i, x_i + \epsilon_i]$ contains $\alpha$
and is thus nonempty.
\vfill\pagebreak
\item[$3\le 1$] Suppose $(x_i)_i\subset\RR$ is Cauchy. Let $a_i = \min\overline{\{x_j : i < j\}}$, $b_i = \max\overline{\{x_j : i < j\}}$.
Then for every $\epsilon > 0$ there is some $N\in\NN$ such that for all $i\ge N$ we have 
$b_i - a_i < \epsilon$. In particular for every $i\in\NN$ there is some
$\sigma(i)\in\NN$ such that $b_{\sigma(i)} - a_{\sigma(i)} < 2^{-i}$.
We then have
\[[a_{\sigma(0)}, b_{\sigma(0)}]\supset[a_{\sigma(1)}, b_{\sigma(1)}]\supset[a_{\sigma(2)}, b_{\sigma(2)}]
\supset\cdots
\]
with $b_{\sigma(i)} - a_{\sigma(i)}\to 0$, so the
intersection $\bigcap_i[a_{\sigma(i)}, b_{\sigma(i)}]$ is nonempty. Pick $x_\infty\in\bigcap_i[a_{\sigma(i)}, b_{\sigma(i)}]$. Now,
$a_{\sigma(i)} < x_\infty < b_{\sigma(i)}$ for all $i\in\NN$, so
\[0 < x_\infty - a_{\sigma(i)} < b_{\sigma(i)} - a_{\sigma(i)}\qquad\text{and}\qquad
a_{\sigma(i)} - b_{\sigma(i)} < x_\infty - b_{\sigma(i)} < 0\]
for all $i\in\NN$. By the squeeze theorem, we have both $a_i\to x_\infty$ and $b_i\to x_\infty$.
At least one of $\{(a_{\sigma(i)})_i, (b_{\sigma(i)})\}$ attains
infinitely many distinct values, let's say $(a_{\sigma(i)})$ does this.
Construct the sequence $(c_i)_i$ to be $(a_{\sigma(i)})_i$
without repeats. This is a subsequence of both $(a_{\sigma(i)})_i$
and $(x_i)_i$. But $a_{\sigma(i)}\to x_\infty$,
so $c_i\to x_\infty$, hence $x_i\to x_\infty$.
\item[$2\le 4$] Let $(x_i)_i\subset\RR$ be strictly increasing and bounded above.
Then $\alpha = \sup_ix_i$ exists. Let $\epsilon > 0$. Then there is some $x_i$
such that $\alpha < x_j + \epsilon$ for all $j > i$,
hence $|\alpha - x_j| < \epsilon$ for all $j > i$,
showing $x_j\to\alpha$.
\item[$4\le 5$] Suppose $(x_i)_i\subset\RR$ is bounded.
Take the subsequence $(x_{\sigma(i)})_i = (\min_{i < j}x_j)_i$.
This sequence is strictly decreasing and bounded below, so it converges
to some $x_\infty$.
\item[$5\le 6$]
Let $f:[a, b]\to\RR$ be continuous.
Take the sequence $(x_i)_i$ that goes
\[a,\quad b,\quad\frac{a + b}{2},\quad\frac{3a + b}{4},\quad\frac{a + 3b}{4},\quad\frac{7a + b}{8},\quad\frac{5a + 3b}{8},\quad\cdots\]
this sequence is dense in $[a, b]$. Since continuous functions map dense subsets to dense subsets,
$f_*([a, b])\cap[f(a), f(b)]$ forms a dense subset of $[f(a), f(b)]$.
Construct the infinite array of subsequences by stipulating that row $n$ of the array consists of the
subsequence of $(x_i)_i$ for which $f(x_i)\in[f(a)/n, f(b)/n]$ for all terms in the $n-1$th row (take row $0$ to be $(x_i)_i$).
By density, the rows of this infinite array are nonempty, in fact each row has infinitely many entries.
Since the terms of the infinite array are bounded, it follows that the leftmost column of the array is bounded. Thus the
leftmost column has a convergent subsequence. By sequential continuity of $f$, the image of this convergent subsequence is itself a convergent
sequence. Due to how the infinite array of subsequences was constructed, though, the image of the limit of the convergent subsequence
has no choice but to be $0$. So the limit of the leftmost column sequence is a $c$ such that $f(c) = 0$.
\item[$6\le 2$] Let $S\subset\RR$ be nonempty and bounded above and suppose
that $\sup S$ does not exist. Let $T$ be the set of all upper bounds
of $S$, and define $f:\RR\to\RR$ to take the values $-1$ on $\RR\setminus T$
and $1$ on $T$. Then $f$ is continuous, but for all $x\in\RR$, we have $f\ne 0$,
thereby falsifying the intermediate value theorem.
So by contraposition, $\sup S$ must exist.
\end{itemize}
Hence, $1$ through $6$ are logically equivalent given $0$.
\end{proof}
\end{document}
